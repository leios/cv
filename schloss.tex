%%%%%%%%%%%%%%%%%%%%%%%%%%%%%%%%%%%%%%%%%
% Medium Length Graduate Curriculum Vitae
% LaTeX Template
% Version 1.1 (9/12/12)
%
% This template has been downloaded from:
% http://www.LaTeXTemplates.com
%
% Original author:
% Rensselaer Polytechnic Institute (http://www.rpi.edu/dept/arc/training/latex/resumes/)
%
% Important note:
% This template requires the res.cls file to be in the same directory as the
% .tex file. The res.cls file provides the resume style used for structuring the
% document.
%
%%%%%%%%%%%%%%%%%%%%%%%%%%%%%%%%%%%%%%%%%

%----------------------------------------------------------------------------------------
%	PACKAGES AND OTHER DOCUMENT CONFIGURATIONS
%----------------------------------------------------------------------------------------

\documentclass[margin, 10pt, a4paper]{res} % Use the res.cls style, the font size can be changed to 11pt or 12pt here

\usepackage{helvet} % Default font is the helvetica postscript font
\usepackage{enumitem}
\usepackage{multicol}
%\usepackage{newcent} % To change the default font to the new century schoolbook postscript font uncomment this line and comment the one above

\setlength{\textwidth}{5.1in} % Text width of the document

\begin{document}

%----------------------------------------------------------------------------------------
%	NAME AND ADDRESS SECTION
%----------------------------------------------------------------------------------------

\moveleft.5\hoffset\centerline{\large\bf James Schloss} % Your name at the top
 
\moveleft\hoffset\vbox{\hrule width\resumewidth height 1pt}\smallskip % Horizontal line after name; adjust line thickness by changing the '1pt'
 
\moveleft.5\hoffset\centerline{Leios Labs LLC} % Your address
\moveleft.5\hoffset\centerline{jrs.schloss@gmail.com}

%----------------------------------------------------------------------------------------

\begin{resume}
\vspace{-6mm}

%----------------------------------------------------------------------------------------
%	EDUCATION SECTION
%----------------------------------------------------------------------------------------

\section{EDUCATION}
{\sl Ph.D.} \hfill 2019\\
Okinawa Institute of Science and Technology, Onna-son, Okinawa, JP

{\sl B.Sc.,} Physics \hfill 2014\\
Auburn University, Auburn, AL, US 

 
%----------------------------------------------------------------------------------------
%	PROFESSIONAL EXPERIENCE SECTION
%----------------------------------------------------------------------------------------
 
\section{RESEARCH EXPERIENCE}

{\sl CEO} \hfill ongoing \\
LeiosLabs LLC

\begin{itemize}[label = {}] \itemsep -2pt % Reduce space between items
\item Open research related to heterogeneous computing, computer graphics, and computational science
\end{itemize}

{\sl Postdoctoral Fellow} \hfill 2020-2021 \\
Primary Investigator: Raffaele Ferrari, MIT

\begin{itemize}[label = {}] \itemsep -2pt % Reduce space between items
\item Developing Heterogeneous computing methods (CPU + GPGPU) for the Climate Machine (CLIMA) project in collaboration with MIT, CalTech, and NPS
\end{itemize}

{\sl JSPS fellow, Quantum Systems} \hfill 2014-2019 \\
Primary Investigator: Thomas Busch, OIST

\begin{itemize}[label = {}] \itemsep -2pt % Reduce space between items
\item Developing GPU computing methods to simulate vortex dynamics in superfluid Bose--Einstein condensates
\end{itemize}

{\sl Research Assistant, Neurobiology Research Unit} \hfill Summer 2015 \\
Primary Investigator: Jeff Wickens, OIST

\begin{itemize}[label = {}] \itemsep -2pt % Reduce space between items
\item Developed an analogue electronic neurosynaptic core for cognitive computing
\end{itemize}

{\sl Research Assistant, Quantum Systems} \hfill Spring 2015 \\
Primary Investigator: Thomas Busch, OIST

\begin{itemize}[label = {}] \itemsep -2pt % Reduce space between items
\item Performed the Chopped RAndom Basis (CRAB) optimum control technique on a rotational ring of strongly correlated ultracold atoms
\end{itemize}

{\sl Research Assistant, Light-Matter Interaction} \hfill Fall 2014 \\
Primary Investigator: S\'ile Nic Chormaic, OIST

\begin{itemize}[label = {}] \itemsep -2pt % Reduce space between items
\item Investigated particle trapping in plasmonic nano-aperture arrays
\end{itemize}

{\sl Los Alamos Computational Physics Student Summer Workshop} \hfill Summer 2014 \\
Primary Investigator: Jerome Daligault, Los Alamos National Labs
\begin{itemize}[label = {}] \itemsep -2pt % Reduce space between items
\item Developed ``Visualization with Blender of One-Component Plasma and Thomas-Fermi Molecular Dynamics Simulations" (VBOTS)
\end{itemize}

{\sl Research Assistant, Plasma Physics} \hfill Summer 2012 -- Summer 2014 \\
Primary Investigator: Edward Thomas, Auburn University

\begin{itemize}[label = {}] \itemsep -2pt % Reduce space between items
\item Developed for the ``Dynamic Exploration of Microparticle clouds Optimized Numerically" (DEMON) simulation code
\end{itemize}

{\sl Research Assistant, Plasma Physics} \hfill Summer 2012 -- Summer 2014\\
Primary Investigators: Stuart Loch and Connor Ballance, Auburn University
\begin{itemize} [label = {}]
\item Performed Monte Carlo simulations of ions and electrons in plasma systems
\end{itemize} 

{\sl Research Assistant, Psychology} \hfill Fall 2011 -- Spring 2012\\
Primary Investigator: Tracy Witte

\begin{itemize} [label = {}]
\item Studied pain tolerance in patients contemplating suicide
\end{itemize} 

%----------------------------------------------------------------------------------------
%	COMMUNITY SERVICE SECTION
%---------------------------------------------------------------------------------------- 

\section{TEACHING EXPERIENCE}

{\sl Technical Seminars, OIST} \hfill{2014-2019}
\begin{itemize}[label = {}]
\item Instructed a number of short, intensive classes on a range of technical topics, including: Programming with python, Data structures and algorithms, Git, Fourier Transforms, Filmmaking for Scientists, Julia, Terminal, GPGPU computing, gnuplot, and bioinformatics.
\end{itemize}

{\sl Teaching Assistant, Physics} \hfill Spring 2013 - Spring 2014 
\begin{itemize}[label = {}]
\item Instructed laboratory sessions for general physics (PHYS 1000) and trigonometry-based physics II (PHYS 1510) at Auburn University.
\end{itemize} 

\section{ACADEMIC WORKS}

{\sl \textbf{Papers:} } \\
  
  \textit{Uncertainty Quantification of Ocean Parameterizations: Application to the K‐Profile‐Parameterization for Penetrative Convection} \\
  Andre Nogueira Souza, GL Wagner, Ali Ramadhan, B Allen, V Churavy, James Schloss, J Campin, Chris Hill, Alan Edelman, John Marshall, G Flierl, Raffaele Ferrari \\
  Journal of Advances in Modeling Earth Systems 12 (12), e2020MS002108
  
  \textit{Controlled creation of three-dimensional vortex structures in Bose-Einstein condensates using artificial magnetic fields} \\
  James Schloss, Peter Barnett, Rashi Sachdeva, Thomas Busch \\
  Phys Rev A 102 (4), 043325
  
 \textit{Non-adiabatic generation of NOON states in a Tonks-Girardeau gas} \\
  J. Schloss, A. Benseny, J. Gillet, J. Swain, Th. Busch \\
  New Journal of Physics 18 (3), 035012
  
  \textit{GPUE: Graphics Processing Unit Gross--Pitaevskii Equation solver} \\
  J Schloss, LJ O'Riordan \\
  Journal of Open Source Software 3 (32), 1037
  
  \textit{Chaotic few-body vortex dynamics in rotating Bose-Einstein condensates} \\
  T Zhang, J Schloss, A Thomasen, LJ O'Riordan, T Busch, A White
  Physical Review Fluids 4 (5), 054701
 
 {\sl \textbf{Awards, Grants, and Fellowships:}} \\
 JSPS KAKENHI Grants-in-aid, JP17J01488 \hfill 2017-2019 \\
 JSPS DC1 Research Fellowship for Young Scientists \hfill 2017-2019\\
 Auburn University Physics Department Undergraduate Research Award \hfill 2014

\newpage
%----------------------------------------------------------------------------------------
%	EXTRA-CURRICULAR ACTIVITIES SECTION
%----------------------------------------------------------------------------------------

\section{PROFESSIONAL SKILLS} 

{\sl \textbf{Languages \& Software:}} 
Julia, C++, CUDA, Fortran, Python, OpenGL, Linux  \\

{\sl \textbf{Software projects:}} \\
Algorithm Archive: www.algorithm-archive.org/ \\
KernelAbstractions, MIT: https://juliagpu.github.io/KernelAbstractions.jl/stable/\\
GPUE, OIST: github.com/GPUE-group/GPUE \\
DEMON simulation code, Auburn University: github.com/AU-PSL/demonsimulationcode \\
VBOTS visualization, Los Alamos National Labs. \\

\section{PUBLIC WORKS}

Arcane Algorithm Archive: algorithm-archive.org \\
LeiosLabs, youtube: youtube.com/user/leiosos \\
LeiosLabs, twitch: twitch.tv/simuleios \\
Loachapoka Elementary science demonstrations, Auburn. \\
OIST high school talks.\\

{\sl Additional details:} \\
President of Computer Science Club at OIST \\
Communications officer, OIST Student Council (2017) \\
IT officer, OIST Student Council (2018)



%----------------------------------------------------------------------------------------

\end{resume}
\end{document}

